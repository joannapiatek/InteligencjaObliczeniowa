\section{Własne funkcje mutacji, krzyżowania i selekcji}

Zadanie polegało na zastąpieniu domyślnych funkcji używanych w pakiecie GA na własne implementacje i porównanie ich działania. Podstawowe zestawienie składa się z najlepszych oraz średnich wyników dla danej populacji w przypadku użycia funkcji wbudowanej oraz własnej. Podczas badań zmieniane są parametry dotyczące badanej funkcji, natomiast pozostałe przyjmują wartości domyślne podane poniżej.\\

Wartości domyślne funkcji użytych w badaniach to kolejno:\\
Rozmiar populacji - 100\\
Liczba iteracji – 50\\
Prawdopodobieństwo krzyżowania – 0.5\\
Prawdopodobieństwo wystąpienia mutacji - 0.1\\
Selekcja elitarnych jednostek – 6\\
\newline

Wyniki wszystkich badań to rezultaty uśrednione po 30 przebiegach.\\
\newline

Do badań została wykorzystana funkcja wielomodalna \textit{branin}.

\subsection{Mutacja}

\subsubsection{Kod źródłowy}



\subsubsection{Wyniki badań}

\begin{figure}[H]
	\centering
	\hspace*{-0.8in}
	\includegraphics[scale = 0.5]{mut_0_1}
	\caption{Wykres dla prawdopodobieństwa mutacji 0.1}  
	\label{rys:mut_0.1} 
\end{figure}

\begin{figure}[H]
	\centering
	\hspace*{-0.8in}
	\includegraphics[scale = 0.5]{mut_0_5}
	\caption{Wykres dla prawdopodobieństwa mutacji 0.5}  
	\label{rys:mut_0.5} 
\end{figure}

\begin{figure}[H]
	\centering
	\hspace*{-0.8in}
	\includegraphics[scale = 0.5]{mut_0_7}
	\caption{Wykres dla prawdopodobieństwa mutacji 0.7}  
	\label{rys:mut_0.7} 
\end{figure}

\begin{figure}[H]
	\centering
	\hspace*{-0.8in}
	\includegraphics[scale = 0.5]{mut_1}
	\caption{Wykres dla prawdopodobieństwa mutacji 1}  
	\label{rys:mut_1} 
\end{figure}



\subsubsection{Wnioski}

\begin{table}[!h]
	\hspace*{-1.5in}
	\centering
	\caption{Wartości średnie i najlepsze osobnika dla domyślnej i własnej funkcji mutacji}
	\label{mut_porownanie}
	\hspace*{-0.4in}
	\begin{tabular}{|c|c|c|c|c|}
		\hline
		\textbf{Prawdopodobieństwo} & \multicolumn{2}{c}{\textbf{Mutacja domyślna}}  & \multicolumn{2}{|c|}{\textbf{Mutacja własna}} \\ \cline{2-5}
		\textbf{mutacji} & Wartość średnia & Najlepszy wynik & Wartość średnia & Najlepszy wynik \\ \hline
		
		0.1 & 5.753710  & 0.398006 & 0.398736 & 0.398687 \\
		0.5 & 31.029570 & 0.401904 & 0.415733 & 0.398201 \\
		0.7 & 38.184020 & 0.404532 & 0.567679 & 0.398926 \\
		1   & 46.920430 & 0.408154 & 0.452637 & 0.400847  \\ \hline      
	\end{tabular}
\end{table}
\subsection{Selekcja}

\subsubsection{Kod źródłowy}


\subsubsection{Wyniki badań}


\begin{figure}[]
	\centering
	\hspace*{-0.8in}
	\includegraphics[scale = 0.5]{sel_1}
	\caption{Wykres przy 1 osobniku elitarnym}  
	\label{rys:sel_1} 
\end{figure}

\begin{figure}[H]
	\centering
	\hspace*{-0.8in}
	\includegraphics[scale = 0.5]{sel_6}
	\caption{Wykres przy 6 osobnikach elitarnych}  
	\label{rys:sel_6} 
\end{figure}

\begin{figure}[H]
	\centering
	\hspace*{-0.8in}
	\includegraphics[scale = 0.5]{sel_20}
	\caption{Wykres przy 20 osobnikach elitarnych}  
	\label{rys:sel_20} 
\end{figure}

\begin{figure}[H]
	\centering
	\hspace*{-0.8in}
	\includegraphics[scale = 0.5]{sel_50}
	\caption{Wykres przy 50 osobnikach elitarnych}  
	\label{rys:sel_50} 
\end{figure}

\subsubsection{Wnioski}

\begin{table}[!h]
	\centering
	\caption{Wartości średnie i najlepsze osobnika dla domyślnej i własnej funkcji selekcji}
	\label{sel_porownanie}
	\begin{tabular}{|c|c|c|c|c|}
		\hline
		\textbf{Selekcja} & \multicolumn{2}{c}{\textbf{Selekcja domyślna}}  & \multicolumn{2}{|c|}{\textbf{Selekcja własna}} \\ \cline{2-5}
		\textbf{elitarna} & Wartość średnia & Najlepszy wynik & Wartość średnia & Najlepszy wynik \\ \hline
		
		1  & 13.660880  & 0.405495 & 22.860800 & 0.414707 \\
		6  & 7.106856 & 0.400895 & 5.544652 & 0.397909 \\
		20 & 5.847374 & 0.397888 & 5.678752 & 0.397977 \\
		50 & 5.42606 & 0.397903 & 4.833812 & 0.397888  \\ \hline      
	\end{tabular}
\end{table}
\subsection{Krzyżowanie}

\subsubsection{Kod źródłowy}



\subsubsection{Wyniki badań}

\begin{figure}[H]
	\centering
	\hspace*{-0.8in}
	\includegraphics[scale = 0.5]{img/zad1/cross_0_2}
	\caption{Wykres dla prawdopodobieństwa krzyżowania 0.2}  
	\label{rys:cross_0_2} 
\end{figure}


\begin{figure}[H]
	\centering
	\hspace*{-0.8in}
	\includegraphics[scale = 0.5]{img/zad1/cross_0_5}
	\caption{Wykres dla prawdopodobieństwa krzyżowania 0.5}  
	\label{rys:cross_0_5} 
\end{figure}


\begin{figure}[H]
	\centering
	\hspace*{-0.8in}
	\includegraphics[scale = 0.5]{img/zad1/cross_0_7}
	\caption{Wykres dla prawdopodobieństwa krzyżowania 0.7}  
	\label{rys:cross_0_7} 
\end{figure}


\begin{figure}[H]
	\centering
	\hspace*{-0.8in}
	\includegraphics[scale = 0.5]{img/zad1/cross_1}
	\caption{Wykres dla prawdopodobieństwa krzyżowania 1}  
	\label{rys:cross_1} 
\end{figure}


\subsubsection{Wnioski}

\begin{table}[!h]
	\centering
	\caption{Wartości średnie i najlepsze osobnika dla domyślnej i własnej funkcji krzyżowania}
	\label{cross_porownanie}
	\hspace*{-0.5in}
	\begin{tabular}{|c|c|c|c|c|}
		\hline
		\textbf{Prawdopodobieństwo} & \multicolumn{2}{c}{\textbf{Krzyżowanie domyśln}}  & \multicolumn{2}{|c|}{\textbf{Krzyżowanie własne}} \\ \cline{2-5}
		\textbf{krzyżowania} & Wartość średnia & Najlepszy wynik & Wartość średnia & Najlepszy wynik \\ \hline
		
		0.2 & 5.697605 & 0.399347 & 6.538159 & 0.440933 \\
		0.5 & 6.973119 & 0.398064 & 7.590295 & 0.457468 \\
		0.7 & 7.753576 & 0.398096 & 8.148286 & 0.464140 \\
		1   & 10.206810 & 0.398485 & 10.375570 & 0.476180  \\ \hline      
	\end{tabular}
\end{table}
\subsection{Główny kod źródłowy}

Poniższe kody źródłowe zawierają definiowanie używanych funkcji uruchamiających algorytmy genetyczne z pakietu \textit{GA}, a następnie uśredniających wyniki ich działania. Pierwszy fragment zawiera implementację z domyślnymi funkcjami mutacji, selekcji i krzyżowania, natomiast w następnym używane są funkcje autorskie. Na trzecim listingu przedstawione jest finalne wywołanie tych funkcji.\\
Warto zaznaczyć, że wynikiem optymalizacji są liczby przeciwne do tych, które powinno się otrzymać. Wynika to z następującej linii kodu w wywołaniu algorytmu \textit{GA}:\\

\begin{lstlisting}[linewidth=10.0cm]
   fitness = function(x) - branin(x[1], x[2])
\end{lstlisting}
\vline

- funkcja podawana jest ze znakiem '-'. W celu odwrócenia znaku wyników otrzymane uśrednione wartości mnożymy przez -1, co można zobaczyć w trzecim listingu.\\

\begin{lstlisting}[linewidth=10.0cm]
   meanRowsMax <- -rowMeans(gaResult$max)
\end{lstlisting}


\newpage

\begin{lstlisting}[linewidth=15.4cm]
# Implementacja funkcji pozwalajacej na zmiane parametrow
meanGA <- function(population,iteration,crosing,mutant,mini,maxi,elit)
{
   xMax <- matrix ( 0 ,iteration,loopRepetitions)   
   xMean <- matrix ( 0 ,iteration,loopRepetitions)  

   # Petla do usredniania wynikow
   for ( i in 1:loopRepetitions )
   {  
	GA <- ga(type = "real-valued",
	fitness = function(x) - branin(x[1], x[2]),
	min = mini, max = maxi, pcrossover = crosing, pmutation = mutant,
	popSize = population, maxiter = iteration, elitism = elit)
	
	xMax[,i] <- GA@summary[,1]
	xMean[,i] <- GA@summary[,2]
	print(GA@solution)
   }
   return(list(max=xMax,min=xMean))
}

\end{lstlisting}


\begin{lstlisting}[linewidth=15.4cm]
# Wersja uzywajaca wlasnych implementacji funkcji (np. mutacji)
meanGaCustom <- function(population,iteration,crosing,mutant,mini,maxi,elit)
{
   xMax <- matrix ( 0 ,iteration,loopRepetitions)   
   xMean <- matrix ( 0 ,iteration,loopRepetitions)  

   # Petla do usredniania wynikow
   for ( i in 1:loopRepetitions )
   {  
	GA <- ga(type = "real-valued",
	fitness = function(x) - branin(x[1], x[2]),
	
	# Wlasne implementacje funkcji odkomentowywane w miare potrzeb
	#mutation = myMutation,
	#selection = mySelection,
	crossover = myCrossover,
	
	min = mini, max = maxi, pcrossover = crosing, pmutation = mutant,
	popSize = population, maxiter = iteration, elitism = elit)
	
	xMax[,i] <- GA@summary[,1]
	xMean[,i] <- GA@summary[,2]
	print(GA@solution)
   }
   return(list(max=xMax,min=xMean,sol=GA@solution))
}
\end{lstlisting}


\begin{lstlisting}[linewidth=15.4cm]
# Ustawienie parametrow
population <- 100 
iteration <- 50
crosing <- 1
mutant <- 0.1
mini <- c(-5,-5)
maxi <- c(15,15)
elit <- 6

# Finalne wywolanie funkcji domyslnej
gaResult <- meanGA(population, iteration, crosing, mutant, 
mini, maxi, elit)
# Wartosci srednie i najlepsze do wykresow
meanRowsMax <- -rowMeans(gaResult$max)
meanRowsMean <- -rowMeans(gaResult$min)

# Finalne wywolanie funkcji z wlasnymi implementacjami 
gaResultCustom <- meanGaCustom(population, iteration, crosing, mutant, 
mini, maxi, elit)
# Wartosci srednie i najlepsze do wykresow
meanRowsMaxCustom <- -rowMeans(gaResultCustom$max)
meanRowsMeanCustom <- -rowMeans(gaResultCustom$min)

# Pobranie sredniej wartosci dla ostatnich iteracji - wyniki koncowe
round(tail(meanRowsMean, n=1), digit=6)
round(tail(meanRowsMax, n=1),  digit=6)
round(tail(meanRowsMeanCustom, n=1), digit=6)
round(tail(meanRowsMaxCustom, n=1), digit=6)

\end{lstlisting}

\newpage
\section{Własne funkcje mutacji, krzyżowania i selekcji}


\subsection{Wstęp}
Zadanie polegało na zastąpieniu domyślnych funkcji używanych w pakiecie GA na własne implementacje i porównanie ich działania. Podstawowe zestawienie składa się z najlepszych oraz średnich wyników dla danej populacji w przypadku użycia funkcji wbudowanej oraz własnej. Podczas badań zmieniane są parametry dotyczące badanej funkcji, natomiast pozostałe przyjmują wartości domyślne podane poniżej.\\

Wartości domyślne funkcji użytych w badaniach to kolejno:\\
Rozmiar populacji - 100\\
Liczba iteracji – 50\\
Prawdopodobieństwo krzyżowania – 0.5\\
Prawdopodobieństwo wystąpienia mutacji - 0.1\\
Selekcja elitarnych jednostek – 6\\
\newline

Wyniki wszystkich badań to rezultaty uśrednione po 30 przebiegach.\\
\newline

\subsection{Funkcja \textit{branin}}
Do badań została wykorzystana funkcja wielomodalna \textit{branin}.\\

Wzór funkcji \textit{branin}:
\begin{figure}[H]
	\centering
	\includegraphics[scale = 1]{branin_formula} 
	\label{rys:branin_formula} 
\end{figure}

Minimum globalne:
\begin{figure}[H]
	\centering
	\includegraphics[scale = 1]{branin_minimum}
	\label{rys:branin_minimum} 
\end{figure}

Poniżej przedstawiono kod źródłowy importujący funkcję \textit{branin} z pakietu \textit{globalOptTests}.\\

\begin{lstlisting}[frame=single]
library(globalOptTests)

branin <- function(x1,x2)
{
   return(goTest(fnName="Branin", par=c(x1,x2)))
}
\end{lstlisting}

\begin{figure}[H]
	\centering
	\includegraphics[scale = 1]{branin_plot}
	\caption{Wykres funkcji \textit{branin}}  
	\label{rys:branin_plot} 
\end{figure}

\vline

\subsection{Główny kod źródłowy}

Poniższe kody źródłowe zawierają definiowanie używanych funkcji uruchamiających algorytmy genetyczne z pakietu \textit{GA}, a następnie uśredniających wyniki ich działania. Pierwszy fragment zawiera implementację z domyślnymi funkcjami mutacji, selekcji i krzyżowania, natomiast w następnym używane są funkcje autorskie. Na trzecim listingu przedstawione jest finalne wywołanie tych funkcji.\\
Warto zaznaczyć, że wynikiem optymalizacji są liczby przeciwne do tych, które powinno się otrzymać. Wynika to z następującej linii kodu w wywołaniu algorytmu \textit{GA}:\\

\begin{lstlisting}[linewidth=10.0cm]
   fitness = function(x) - branin(x[1], x[2])
\end{lstlisting}
\vline

- funkcja podawana jest ze znakiem '-'. W celu odwrócenia znaku wyników otrzymane uśrednione wartości mnożymy przez -1, co można zobaczyć w trzecim listingu.\\

\begin{lstlisting}[linewidth=10.0cm]
   meanRowsMax <- -rowMeans(gaResult$max)
\end{lstlisting}


\newpage

\begin{lstlisting}[linewidth=15.4cm]
# Implementacja funkcji pozwalajacej na zmiane parametrow
meanGA <- function(population,iteration,crosing,mutant,mini,maxi,elit)
{
   xMax <- matrix ( 0 ,iteration,loopRepetitions)   
   xMean <- matrix ( 0 ,iteration,loopRepetitions)  

   # Petla do usredniania wynikow
   for ( i in 1:loopRepetitions )
   {  
	GA <- ga(type = "real-valued",
	fitness = function(x) - branin(x[1], x[2]),
	min = mini, max = maxi, pcrossover = crosing, pmutation = mutant,
	popSize = population, maxiter = iteration, elitism = elit)
	
	xMax[,i] <- GA@summary[,1]
	xMean[,i] <- GA@summary[,2]
	print(GA@solution)
   }
   return(list(max=xMax,min=xMean))
}

\end{lstlisting}


\begin{lstlisting}[linewidth=15.4cm]
# Wersja uzywajaca wlasnych implementacji funkcji (np. mutacji)
meanGaCustom <- function(population,iteration,crosing,mutant,mini,maxi,elit)
{
   xMax <- matrix ( 0 ,iteration,loopRepetitions)   
   xMean <- matrix ( 0 ,iteration,loopRepetitions)  

   # Petla do usredniania wynikow
   for ( i in 1:loopRepetitions )
   {  
	GA <- ga(type = "real-valued",
	fitness = function(x) - branin(x[1], x[2]),
	
	# Wlasne implementacje funkcji odkomentowywane w miare potrzeb
	#mutation = myMutation,
	#selection = mySelection,
	crossover = myCrossover,
	
	min = mini, max = maxi, pcrossover = crosing, pmutation = mutant,
	popSize = population, maxiter = iteration, elitism = elit)
	
	xMax[,i] <- GA@summary[,1]
	xMean[,i] <- GA@summary[,2]
	print(GA@solution)
   }
   return(list(max=xMax,min=xMean,sol=GA@solution))
}
\end{lstlisting}


\begin{lstlisting}[linewidth=15.4cm]
# Ustawienie parametrow
population <- 100 
iteration <- 50
crosing <- 1
mutant <- 0.1
mini <- c(-5,-5)
maxi <- c(15,15)
elit <- 6

# Finalne wywolanie funkcji domyslnej
gaResult <- meanGA(population, iteration, crosing, mutant, 
mini, maxi, elit)
# Wartosci srednie i najlepsze do wykresow
meanRowsMax <- -rowMeans(gaResult$max)
meanRowsMean <- -rowMeans(gaResult$min)

# Finalne wywolanie funkcji z wlasnymi implementacjami 
gaResultCustom <- meanGaCustom(population, iteration, crosing, mutant, 
mini, maxi, elit)
# Wartosci srednie i najlepsze do wykresow
meanRowsMaxCustom <- -rowMeans(gaResultCustom$max)
meanRowsMeanCustom <- -rowMeans(gaResultCustom$min)

# Pobranie sredniej wartosci dla ostatnich iteracji - wyniki koncowe
round(tail(meanRowsMean, n=1), digit=6)
round(tail(meanRowsMax, n=1),  digit=6)
round(tail(meanRowsMeanCustom, n=1), digit=6)
round(tail(meanRowsMaxCustom, n=1), digit=6)

\end{lstlisting}

\newpage
\subsection{Mutacja}

\subsubsection{Kod źródłowy}



\subsubsection{Wyniki badań}

\begin{figure}[H]
	\centering
	\hspace*{-0.8in}
	\includegraphics[scale = 0.5]{mut_0_1}
	\caption{Wykres dla prawdopodobieństwa mutacji 0.1}  
	\label{rys:mut_0.1} 
\end{figure}

\begin{figure}[H]
	\centering
	\hspace*{-0.8in}
	\includegraphics[scale = 0.5]{mut_0_5}
	\caption{Wykres dla prawdopodobieństwa mutacji 0.5}  
	\label{rys:mut_0.5} 
\end{figure}

\begin{figure}[H]
	\centering
	\hspace*{-0.8in}
	\includegraphics[scale = 0.5]{mut_0_7}
	\caption{Wykres dla prawdopodobieństwa mutacji 0.7}  
	\label{rys:mut_0.7} 
\end{figure}

\begin{figure}[H]
	\centering
	\hspace*{-0.8in}
	\includegraphics[scale = 0.5]{mut_1}
	\caption{Wykres dla prawdopodobieństwa mutacji 1}  
	\label{rys:mut_1} 
\end{figure}



\subsubsection{Wnioski}

\begin{table}[!h]
	\hspace*{-1.5in}
	\centering
	\caption{Wartości średnie i najlepsze osobnika dla domyślnej i własnej funkcji mutacji}
	\label{mut_porownanie}
	\hspace*{-0.4in}
	\begin{tabular}{|c|c|c|c|c|}
		\hline
		\textbf{Prawdopodobieństwo} & \multicolumn{2}{c}{\textbf{Mutacja domyślna}}  & \multicolumn{2}{|c|}{\textbf{Mutacja własna}} \\ \cline{2-5}
		\textbf{mutacji} & Wartość średnia & Najlepszy wynik & Wartość średnia & Najlepszy wynik \\ \hline
		
		0.1 & 5.753710  & 0.398006 & 0.398736 & 0.398687 \\
		0.5 & 31.029570 & 0.401904 & 0.415733 & 0.398201 \\
		0.7 & 38.184020 & 0.404532 & 0.567679 & 0.398926 \\
		1   & 46.920430 & 0.408154 & 0.452637 & 0.400847  \\ \hline      
	\end{tabular}
\end{table}
\subsection{Selekcja}

\subsubsection{Kod źródłowy}


\subsubsection{Wyniki badań}


\begin{figure}[]
	\centering
	\hspace*{-0.8in}
	\includegraphics[scale = 0.5]{sel_1}
	\caption{Wykres przy 1 osobniku elitarnym}  
	\label{rys:sel_1} 
\end{figure}

\begin{figure}[H]
	\centering
	\hspace*{-0.8in}
	\includegraphics[scale = 0.5]{sel_6}
	\caption{Wykres przy 6 osobnikach elitarnych}  
	\label{rys:sel_6} 
\end{figure}

\begin{figure}[H]
	\centering
	\hspace*{-0.8in}
	\includegraphics[scale = 0.5]{sel_20}
	\caption{Wykres przy 20 osobnikach elitarnych}  
	\label{rys:sel_20} 
\end{figure}

\begin{figure}[H]
	\centering
	\hspace*{-0.8in}
	\includegraphics[scale = 0.5]{sel_50}
	\caption{Wykres przy 50 osobnikach elitarnych}  
	\label{rys:sel_50} 
\end{figure}

\subsubsection{Wnioski}

\begin{table}[!h]
	\centering
	\caption{Wartości średnie i najlepsze osobnika dla domyślnej i własnej funkcji selekcji}
	\label{sel_porownanie}
	\begin{tabular}{|c|c|c|c|c|}
		\hline
		\textbf{Selekcja} & \multicolumn{2}{c}{\textbf{Selekcja domyślna}}  & \multicolumn{2}{|c|}{\textbf{Selekcja własna}} \\ \cline{2-5}
		\textbf{elitarna} & Wartość średnia & Najlepszy wynik & Wartość średnia & Najlepszy wynik \\ \hline
		
		1  & 13.660880  & 0.405495 & 22.860800 & 0.414707 \\
		6  & 7.106856 & 0.400895 & 5.544652 & 0.397909 \\
		20 & 5.847374 & 0.397888 & 5.678752 & 0.397977 \\
		50 & 5.42606 & 0.397903 & 4.833812 & 0.397888  \\ \hline      
	\end{tabular}
\end{table}
\subsection{Krzyżowanie}

\subsubsection{Kod źródłowy}



\subsubsection{Wyniki badań}

\begin{figure}[H]
	\centering
	\hspace*{-0.8in}
	\includegraphics[scale = 0.5]{img/zad1/cross_0_2}
	\caption{Wykres dla prawdopodobieństwa krzyżowania 0.2}  
	\label{rys:cross_0_2} 
\end{figure}


\begin{figure}[H]
	\centering
	\hspace*{-0.8in}
	\includegraphics[scale = 0.5]{img/zad1/cross_0_5}
	\caption{Wykres dla prawdopodobieństwa krzyżowania 0.5}  
	\label{rys:cross_0_5} 
\end{figure}


\begin{figure}[H]
	\centering
	\hspace*{-0.8in}
	\includegraphics[scale = 0.5]{img/zad1/cross_0_7}
	\caption{Wykres dla prawdopodobieństwa krzyżowania 0.7}  
	\label{rys:cross_0_7} 
\end{figure}


\begin{figure}[H]
	\centering
	\hspace*{-0.8in}
	\includegraphics[scale = 0.5]{img/zad1/cross_1}
	\caption{Wykres dla prawdopodobieństwa krzyżowania 1}  
	\label{rys:cross_1} 
\end{figure}


\subsubsection{Wnioski}

\begin{table}[!h]
	\centering
	\caption{Wartości średnie i najlepsze osobnika dla domyślnej i własnej funkcji krzyżowania}
	\label{cross_porownanie}
	\hspace*{-0.5in}
	\begin{tabular}{|c|c|c|c|c|}
		\hline
		\textbf{Prawdopodobieństwo} & \multicolumn{2}{c}{\textbf{Krzyżowanie domyśln}}  & \multicolumn{2}{|c|}{\textbf{Krzyżowanie własne}} \\ \cline{2-5}
		\textbf{krzyżowania} & Wartość średnia & Najlepszy wynik & Wartość średnia & Najlepszy wynik \\ \hline
		
		0.2 & 5.697605 & 0.399347 & 6.538159 & 0.440933 \\
		0.5 & 6.973119 & 0.398064 & 7.590295 & 0.457468 \\
		0.7 & 7.753576 & 0.398096 & 8.148286 & 0.464140 \\
		1   & 10.206810 & 0.398485 & 10.375570 & 0.476180  \\ \hline      
	\end{tabular}
\end{table}
\subsection{Badania przy zmianie dwóch parametrów jednocześnie}

\begin{figure}[H]
	\centering
	\hspace*{-0.8in}
	\includegraphics[scale = 1]{el_mut}
	\caption{Wykres dla własnej funkcji mutacji i selekcji elitarnej}  
	\label{rys:el_mut} 
\end{figure}

Na wykresie widać, że skrajne wartości parametrów powodują gorsze wyniki. Należy więc wybierać uśrednione wartości.

\newpage